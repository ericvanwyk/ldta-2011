\section{Rascal, a DSL for meta-programming}

%\def\rascal{\textsc{Rascal}\xspace}
\def\oberon{\textsc{Oberon-0}\xspace}

% todo: use rascal.sty
\def\rcode#1{\texttt{#1}}

\subsection{Introduction}

\noindent \Rascal is a domain-specific language for
meta-programming. It has been designed to deal with software languages
in the broadest sense of the word. Its application areas include the
development of DSLs, reverse engineering and reengineering of (legacy)
software systems and software repository mining. \Rascal features
powerful domain-specific language constructs to make meta programming
more effective:
\begin{itemize}
\item Integrated syntax definition: context-free grammars can be
  declared as part of ordinary \Rascal code; the parse trees defined
  by such grammars are first-class values, and non-terminals are
  first-class types.
\item Built-in data types: apart from the standard data types (int,
  bool, string etc.), \Rascal supports sets, relations, maps, lists,
  algebraic data types (ADTs), parse trees, and source
  locations. Every value in Rascal can be used in pattern
  matching. This includes concrete syntax trees produced by \Rascal's
  grammar.
\item Constructs for analysis and transformation: sets, relations,
  maps and lists can be used in comprehensions. The visit-statements
  allows for structur-shy traversal and rewriting of abstract or
  concrete syntax trees.
\end{itemize}
\Rascal is \textit{programming language}; it eschews implicit
features. The language design is informed by learnability and
debuggability considerations. This means there is little ``magic'' in
\Rascal. For instance, control-flow is completely straightforward,
unlike, for instance, in languages based on attribute evaluation or
term rewriting~\cite{...}. \Rascal adheres to the
what-you-see-is-what-you-get (WYSIWYG) principle.

\subsubsection{Overview of the \Rascal \oberon implementation}

\noindent \Rascal allows computer languages to be implemented in a modular
fashion, through the simultaneous extension of syntax definition
(grammars), algebraic data types (ADTs) and functions operating on
data conforming to these data types. The \oberon implementation
consists of four layers corresponding to each language level. An
overview of the architecture is shown in Table~\ref{TBL:rascalArch}.

\begin{table}
\begin{center}\small
\begin{tabular}{|l||l|l|l|}\hline
   &  Syntax & Data Type & Function \\\hline\hline
L1 &  \textbf{grammar} & \textbf{AST}, \textbf{scope} & 
\textbf{bind}, \textbf{check}, \textbf{format}, \textbf{compile} \\\hline 
L2 & +grammar & +AST & +bind, +check, +format, \textbf{desugar} \\\hline
L3 & +grammar & +AST, +scope & +bind, +check, +format, +compile, \textbf{lift} \\\hline
L4 & +grammar & +AST & +bind, +check, +format, +compile \\\hline
\end{tabular}
\end{center}
\caption{Layered architecture of the \Rascal \oberon implementation;
  ``+'' indicates extension, \textbf{bold} indicates introduction.\label{TBL:rascalArch}}
\end{table}

The table show three columns, corresponding to syntax definition,
(algebraic) data types and functions. Each row corresponds to a
language level. Level L1 represents the base case: all relevant
syntax, data types and functions are introduced here. In the
subsequent levels any of these types and functions may have to be
extended. For instance, in L2, where the FOR and CASE statements are
introduced, the grammar and AST definition of L1 are extended with new
alternatives to deal with these constructs. Similarly, the functions
bind, check and format are extended. However, the compile function is
not extended; instead FOR and CASE statements are desugared to L1
constructs. Note also that the scope data type used during name
analysis is reused as is. This data type only needs to be extended
when procedures are introduced in L3. In L3 we introduce a lift
function to move nested procedures to top level.

Extension in \Rascal works through the \texttt{extend} construct. This
constructs works similar to module import, except that it allows the
extension of recursively defined abstractions, in this case ADTs and
functions. This leads to a mechanism similar to inheritance in
object-oriented languages~\cite{Cook?}.  For grammars and ADTs the
definitions of the extended module and the extending module are simply
merged. Function extension, on the other hand, depends on a particular
style of writing \Rascal functions, in a way similar to term rewriting
rules. An example may illustrate the approach. For instance the
following rule implements the \texttt{check} function for
type-checking WHILE statements:
\begin{quote}
\begin{rascal}
set[Message] check(whileDo(c, b)) =
                 checkCond(c) + checkBody(b);
\end{rascal}
\end{quote}
It looks like an ordinary \Rascal function definition except that it
uses ``= $\langle$\textit{Expression}$\rangle$'' instead of a regular
function body using $\{\}$. Moreover, the function uses pattern
matching in its signature to dispatch on the WHILE AST constructor;
this style of function definition hence is called
\textit{pattern-based dispatch}.  The complete implementation of the
\texttt{check} function consists of multiple such rules, one for each
AST constructor. Such rules may be distributed over different modules
to obtain extensibility. For instance, in L3 where procedure
definitions and procedure calls are introduced, the \texttt{check}
function is extended by adding an implementation rule dispatching on
the AST constructor for procedure calls. In the \oberon implementation
\texttt{bind}, \texttt{check}, \texttt{format}, and \texttt{compile}
are modularized using pattern-based dispatch.

For syntax and abstract syntax the mechanism is similar. In higher
layers (i.e., L2, L3, and L4), extension modules add alternatives to
both grammars and ADTs. Note that for grammars this requires a general
parsing algorithm, since only the full class of context-free languages
is closed under union~\cite{?}. In L3 this mechanism is used to extend
the scope data type to deal with nested scopes. In L1 this data type
is defined as follows:
\begin{quote}
\begin{rascal}
data NEnv = scope(map[Ident, Decl] env);
\end{rascal}
\end{quote}
In other words the the name environment consists of a plain map from
identifiers to declarations. By wrapping this map in an ADT
constructor, the type \texttt{NEnv} becomes extensible. In L3 it is
extended as follows:
\begin{quote}
\begin{rascal}
data NEnv = nest(map[Ident, Decl] env, NEnv parent);
\end{rascal}
\end{quote}
The new alternative (\texttt{nest}) also consists of a
name-declaration map, but also has an additional parent
\texttt{NEnv}. 

\subsection{Scanning and parsing}

\noindent \Rascal's syntax definitions are backed by a scannerless, general
parsing algorithm based GLL~\cite{JohnstoneScott11}. Scannerless
parsing solves the problem of modular lexical syntax. To deal with
(lexical) ambiguities, \Rascal has built-in support for longest-match
and keyword reservation. Consequently, each language level may
introduce additional reserved keywords, that are not reserved in
previous levels. For instance, the ``FOR'' keyword, introduced in L2,
is not reserved in L1

Grammars are annotated with constructor names to obtain automatic
mapping of parse trees to ASTs. The \Rascal standard library function,
\texttt{implode}, converts and concrete syntax tree into an AST, given
an ADT definition of the abstract syntax. Implode uses the
\textit{reified ADT} as a recipe. Reified types are reflective value
representations of \Rascal types. The information in the reified type
is used by implode to decide how an concrete tree must be mapped to an
AST. For instance, some lexical parse tree nodes are mapped to
(native) integers if this is dictated by the abstract syntax ADT.

The statement syntax of \oberon L1 is defined as follows:
\begin{quote}
\begin{rascal}
syntax Statement 
  = assign: Ident ":=" Expression
  | ifThen: "IF" Expression "THEN" 
               {Statement ";"}+  ElsIfPart* ElsePart? "END"
  | whileDo: "WHILE" Expression "DO" {Statement ";"}+ "END"
  | skip: ;
\end{rascal}
\end{quote}
Each alternative is labeled with a constructor name that corresponds
to a constructor in the abstract syntax, which (for statements) is
defined as follows:
\begin{quote}
\begin{rascal}
data Statement 
  = assign(Ident var, Expression exp)
  | ifThen(Expression condition, list[Statement] body, 
       list[ElsIf] elseIfs, list[Statement] elsePart)
  | whileDo(Expression condition, list[Statement] body)
  | skip()
  ;
\end{rascal}
\end{quote}
In the abstract syntax ordinary \Rascal lists are used for regular
operators.  The \Rascal parser annotates parse trees with source
locations (a native datatype in Rascal). The implode propagates these
locations to the AST so that meaningful error messages can be given
during later processing.

Scannerless means that there is no separate tokenization phase: both
lexical and context-free syntax are described using the same grammar
formalism. This makes it, for instance, trivially easy to support
\oberon's nested comments:
\begin{quote}
\begin{rascal}
lexical Comment = "(*" CommentElt* "*)" ;
lexical CommentElt = CommentChar+ | Comment ;
lexical CommentChar = ![*(] | [*] !>> [)] | [(] !>> [*];
\end{rascal}
\end{quote}
In essence, this is just context-free syntax, except the
\texttt{lexical} keyword indicates that no layout is allowed between
symbols. The CommentChar non-terminal captures all characters except
``*'' and ``('', and uses follow restrictions (``\texttt{!>>}'') to
allow both ``*'' and ``('' if they are not followed by ``)'' and ``*''
respectively.

\subsubsection{Formatting}

\noindent Formatting in \Rascal consists of mapping (abstract) syntax
trees to Box constructs~\cite{Box}. The resulting Box expressions
describes how elements should be layed out, e.g., horizontally,
vertically, indented or aligned, which font-style to use, and how
adjacent elements should be spaced relative to one another.

As an example, consider the formatting of the WHILE statement:
\begin{quote}
\begin{rascal}
Box stat2box(whileDo(c, b)) = V([
   H([KW(L("WHILE")), exp2box(c), KW(L("DO"))])[@hs=1],
      I([V(hsepList(b, ";", stat2box))]),
   KW(L("END"))
]);  
\end{rascal} 
\end{quote}
Again this function uses pattern-based dispatch to match the AST
constructor for WHILE statements. The result of the function is a Box
expression (which is just an ADT in \Rascal). The result dictates that
the WHILE and DO are both keywords (KW), and the condition \irascal{c}
is formatted using the function \irascal{exp2box}. All three sub-boxes
should be laid out horizontally (H). Finally, the spacing between the
elements should be 1, as indicated by the annotation \texttt{@hs=1}.
The body \irascal{b} of the WHILE statement should be indented one
level (I), and the statements should be placed vertically. The helper
function \irascal{hsepList} is used to correctly combine separated
lists where the separator (``;'') should be horizontally combined with
an element; it is parameterized by the a function to convert each
element of the list to a Box expression (i.c., \irascal{stat2box}).
Finally, the three sub-boxes are wrapped in a V box so that the
header, body and END keyword are placed vertically.

A common problem with pretty-printing is the (re)insertion of
parentheses in binary expressions according to the precedence and
associativity rules of the grammar. In \Rascal this can be solved
using reified types. Above, we described how \irascal{implode} used
the reified type of the abstract syntax ADT to guide the mapping of
parse trees to ASTs. In this case we proceed the other way around: we
use the reified type of the \textit{grammar} to obtain precedence and
associativity information. Since all non-terminals are first-class
types in \Rascal, obtaining the reified type of a non-terminal gives
us the complete grammar. 

To illustrate how this works, consider the following formatting rule
for multiplication expressions:
\begin{quote}
\begin{rascal}
Box exp2box(p:mul(lhs, rhs)) = 
    H([exp2box(p, lhs), L("*"), exp2box(p, rhs)])[@hs=1];
\end{rascal}
\end{quote}
This rule lays out expressions horizontally. But instead of calling
directly the unary function \irascal{exp2box} on \irascal{lhs} and
\irascal{rhs}, there is an intermediate call to \irascal{exp2box} with
two arguments; in both cases the first argument is \irascal{p} which
is bound to the current expression using \texttt{p:}. This parent
expression is used to decide whether to insert parentheses or not:
\begin{quote}
\begin{rascal}
Box exp2box(Expression parent, Expression kid) =
    parens(PRIOS, parent, kid, exp2box(kid), parenizer);

Box parenizer(Box box) = H([L("("), box, L(")")])[@hs=0];
\end{rascal}
\end{quote}
In this example, the global constant \irascal{PRIOS} contains
the precedence information obtained from the grammar. The
\irascal{parens} function checks if the occurence of \irascal{kid}
directly below \irascal{parent} requires parentheses, and if so, calls
\irascal{parenizer} on the Box expression resulting from
\irascal{exp2box(kid}). The function \irascal{parenizer} simply
horizontally wraps the argument with parentheses.


\subsection{Name analysis}

\noindent Name analysis consists of a traversal of the AST that does
two things at the same time:
\begin{enumerate}
\item Annotate use sites of variables, types and constants with the
  their declarations.
\item Maintain a set of error messages, indicating problems such as
  undeclared identifiers. 
\end{enumerate}
These two concerns are implemented in functions \irascal{bindModule},
\irascal{bindStat}, \irascal{bindExp} etc. They carry around a
\irascal{NEnv} environment which contains the names that are currently
in scope (cf. above for its definition). If an identifier is
encountered, it is looked up in the environment and, if found, the
identifier is annotated using a \Rascal annotation representing the
declaration.

In order to both annotate ASTs and return a set of error messages the
\irascal{bind} family of functions returns a tuple containing an AST
node and a set of error messsages. This type is defined as the
following \Rascal alias:
\begin{quote}
\begin{rascal}
alias Bound[&T] = tuple[&T ast, set[Message] errs];
\end{rascal}
\end{quote}

As an example, consider the binding of the IF-statement, which is
shown below:
\begin{quote}\small
\begin{rascal}
Bound[Statement] bindStat(s:ifThen(c, b, eis, e), 
                     NEnv nenv, set[Message] errs) {
    <s.condition, errs> = bindExp(c, nenv, errs);
    <s.body, errs> = bindStats(b, nenv, errs);
    s.elseIfs = for (ei <- eis) {
      <ei.condition, errs> = bindExp(ei.condition, nenv, errs);
      <ei.body, errs> = bindStats(ei.body, nenv, errs);
      append ei;
    }
    <s.elsePart, errs> = bindStats(e, nenv, errs);
    return <s, errs>;
}
\end{rascal}
\end{quote}
The function heavily uses \Rascal's destructuring assignment to
simultaneously replace parts of the AST and update the \irascal{errs}
variable. For instance, the first statements invokes \irascal{bindExp}
to bind the condition of the IF-statement; the resulting annotated
expression is inserted in place of the original condition
(\irascal{s.condition}). The reason for this idiom is that it is
impossible to simply rebuild the tree, because then the source
location annotations would be lost. The for-loop folds over the list
of ELSIF's while at the same time (possibly) updating the set
\irascal{errs}. The final result is the annotated IF-node
(\irascal{s}) together with the \irascal{errs}.

The annotation of identifiers is implemented using the following
function: 
\begin{quote}\small
\begin{rascal}
Bound[Ident] bindId(Ident x, NEnv nenv, set[Message] errs) {
  if (isVisible(nenv, x))
    return <x[@decl=getDef(nenv, x)], errs>;  
  if (x.name in {"TRUE", "FALSE"})
    return <x[@decl=trueOrFalse(x.name == "TRUE")], errs>;
  return <x, errs + { undefIdErr(x@location) }>;
}
\end{rascal}
\end{quote}
This function checks the name environment (\irascal{nenv}) if the
identifier \irascal{x} is currently visible. If so, then the
\irascal{x} is annotated with its definition
(\texttt{x[@decl=...]}). Otherwise, if \irascal{x} represents that
boolean constants TRUE or FALSE, it is annotated accordingly. If none
of the above holds, the identifier is undeclared and an error is
produced, containing the source location of \irascal{x}.

\subsection{Type checking}

\noindent Similar to the name-analysis, type checking is an abstract
interpretation of the AST, computing a set of error messages. Unlike
in the case of name-analysis, however, the AST is not annotated with
further information; all required annotations are assumed to be set
during name analysis. This simplifies type checking considerably,
since all required information is local to a certain AST node.

As an example, consider the following code to check procedure calls:
\begin{quote}
\begin{rascal}
set[Message] check(s:call(f, as)) {
  if (!(f@decl is proc)) 
    return { notAProcErr(f@location) };
  if (size(as) != arity(f@decl.formals)) 
    return { argNumErr(s@location) };
  errs = {}; i = 0;
  for (frm <- f@decl.formals, n <- frm.names) {
    errs += checkFormal(n, as[i], frm.hasVar);
    i += 1;
  }
  return ( errs | it + check(a) | a <- as);
}
\end{rascal}
\end{quote}
First the declaration of the called procedure is retrieved in order to
check whether \irascal{f} actually represents a procedure. If not, an
error is returned. Second, list of formal parameters is obtained from
the \irascal{decl} annotations and the arity of the procedure is
checked. Third, now we can assume that the number of actual parameters
is correct, each argument is checked against its corresponding formal
parameter using \irascal{checkFormal}. Finally each actual parameter
is checked individually for type errors. The result of this function
is the union of the set of errors collected in the for-loop and the
union of all sets of errors returned for each actual argument.

\subsection{Source-to-source transformation}

\noindent \Rascal features builtin support for structure-shy traversal
of data structures using the \rcode{visit} statement. Visit works like
a traditional case statement, only cases are matched at arbitrary
depth in a data structure. There are 6 builtin strategies to control
the traversal order. Visited nodes maybe replaced, thereby rewriting
the tree, similar to traversal rules in \cite{ASF+SDF} and rewrite
strategies in \cite{Stratego}.


The desugaring of FOR-loops and CASE-statements both heavily depend on
the visit-statement. For instance, the desugaring of the
for-statements introduced in L2 is shown in
Listing~\cite{LST:for2while}. The function \irascal{for2while} visits
a list of statements. First FOR-loops without a BY-clause are
normalized to have a BY-clause of 1 (\irascal{nat(1)}). Second, if a
BY-clause uses a symbolic constant, it is replaced with the integer
the constant evaluates to. Then, in the thrid and fourth case,
FOR-loops are substituted for equivalent L1 \oberon nodes. Since a
single FOR-loop is desugared to a sequence of statements, we use a
temporary AST constructor, \irascal{begin}, to allow inserting
multiple statements in place of one. The \irascal{begin} nodes are
spliced into their surrounding context in a later phase. Note that the
use of the \rcode{innermost} keyword ensures that FOR-loops are first
normalized and only after that will be eventually desugared.

\begin{listing}[t]
\begin{quote}
\begin{rascal}
public list[Statement] for2while(list[Statement] stats) {
 return innermost visit (stats) {
  case forDo(n, f, t, [], b) => forDo(n, f, t, [nat(1)], b)
  case forDo(n, f, t, [lookup(x)], b) => forDo(n, f, t, [c], b)
           when x@decl is const, c:nat(_) := eval(x)
  case forDo(n, f, t, [nat(x)], b) => 
           begin([assign(n, f), whileDo(leq(lookup(n), t), 
              [b, assign(n, add(lookup(n), nat(x)))])]) 
  case forDo(n, f, t, [neg(nat(x))], b) => 
           begin([assign(n, f), whileDo(geq(lookup(n), t), 
              [b, assign(n, sub(lookup(n), nat(x)))])]) 
 }
}
\end{rascal}
\end{quote}
\caption{Desugaring \oberon FOR-loops to WHILE-loops in
  \Rascal\label{LST:for2while}} 
\end{listing}

The desugaring of the CASE-statement to nested IF-statements is
implemented in a similar way.


\subsection{Code generation}


\subsubsection{Unique names}

\noindent To ensure that there will be no name collisions after
procedures, types and constants are lifted, identifiers referring to
such entities are first ``uniqified''. For names in declarations, the
source location's offset is used to make the name unique\footnote{This
  trivially gives uniqueness, since no two identifiers can occupy the
  same offset in a source file.}. To rename use sites of
identifiers, the location of its declaration is obtained from the
name-binding \irascal{decl} annotation.

The \irascal{rename} function is implementation is shown below:
\begin{quote}
\begin{rascal}
Module rename(Module m, NEnv global) {
  return visit (m) {
    case constDecl(x, e) => constDecl(prime(x), e)
    case typeDecl(x, t) => typeDecl(prime(x), t)
    case proc(x, fs, ds, b, f2) => proc(prime(x), fs, ds, b, f2)
    case call(x, as) => call(prime(x, l)
       when proc(l, _) := x@decl, !isDefined(global, x)
    case user(x) => user(prime(x, l))
      when \type(l, _) := x@decl
    case lookup(x) => Lookup(prime(x, l))
      when const(l, _) := x@decl
  };
}
\end{rascal}
\end{quote}
The \irascal{rename} performs a single traversal over the AST and
inserts \irascal{prime}d names into declarations and use sites. For
the use sites the locations of and identifiers declaration is used.

\subsubsection{Lifting procedures}

\noindent Standard C does not support nested functions. To simplify
the code generation, \oberon procedures are first lifted to the
top-level. Below is the function that lifts nested procedures and
produces a flat list of procedure declarations:
\begin{quote}
\begin{rascal}
public list[Procedure] liftProc(Procedure proc) {
  newProcs = ( [] | it + liftProc(p) | p <- proc.decls.procs );
  proc.decls.procs = [];
  return newProcs + [proc];
}
\end{rascal}
\end{quote}
...
A similar procedure is applied for type and constant declarations. 

\subsubsection{Generating C-code}

\noindent Although it would have been possible to define an abstract
syntax for C and then use \Rascal's Box formatting to obtain a textual
representation suitable for compiling to machine code, we have instead
opted for a simpler, more pragmatic approach using string
templates. In \Rascal string literals can be interpolated with
expressions, conditional statements (if) and loops (for, while,
do-while). This makes string templates very convenient for generating
code. Moreover, using the explicit margin markers (single quote '),
such templates are convenient to write and the result will be
automatically indented. For instance, consider the code to generate
code for WHILE loops and IF statements a C while-loop:
\begin{quote}
\begin{rascal}
str stat2c(whileDo(c, b)) = "while (<exp2c(c)>) {
                            '  <stats2c(b)>
                            '}";

str stat2c(ifThen(c, b, ei, ep)) = 
                            "if (<exp2c(c)>) {
                            '  <stats2c(b)>
                            '}<for (<ec, eb> <- ei) {>
                            'else if (<exp2c(ec)>) {
                            '  <stats2c(eb)>
                            '}<}>
                            '<if (ep != []) {>
                            'else {
                            '  <stats2c(ep)>
                            '}<}>";
\end{rascal}
\end{quote}
In both string templates the statements within statement blocks will
be indented relative to the margin. All whitespace to the left of the
margin is discarded.


% combination of tasks and sublanguages
\subsection{Artifacts}

\subsection{Observations}

Not yet good enough:
- Modularity: not there yet completely; overriding + super
  Example: in L4 the AST nodes for assign/lookup change. This means we
  either have to normalize old AST nodes to new ones and duplicate
  code. Would like to add functionality for the new constructors
  only. (Hmm). Defaults, and explicit extension hooks. :-(

- abstract interpretations (binding/checking); manually thread
environments, and reconstruct tree + errors in return value
tuple. Moreover, must be very careful, not to accidentally discard
location information: e.g. bind(assign(x, e)) = assign(bind(x),
bind(e)) does not work.  Dynamic variables for the set of error
messages. Why not visit for annotation? 1) it is hard to know in
advance whether our builtin strategies are right for the task, 2)
currently, it is not extensible. Down-up strategy?

- Formatting:
essentially a fold. Cannot use visit: type preserving. Need support
for catamorphisms + defaults.


The essence of folds: name analysis, type checking, pretty printing,
and code generation all have been manually implemented as recursive
folds over ASTs. Caveat: depends on language if this is possible; for
\oberon it works well.


Good:

- IDE support
- Grammar formalism
- flexibility
- auto-indenting string templates
- pattern-based dispatch good for modularity
- transformation

\subsection{Conclusions}
